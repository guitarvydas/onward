\documentclass{acmart}

\title{Example Paper Title}
\author{Paul Tarvydas}
\authornote{Both authors contributed equally to this work.}
\orcid{0000-0000-0000-0000}
\affiliation{%
  \institution{Institution1}
  \streetaddress{Street1}
  \city{City1}
  \state{State1}
  \country{Country1}
  \postcode{Zip1}
}
\email{ptarvydas@institution1.edu}

\author{Zac Nowicki}
\authornotemark[1]
\orcid{0000-0000-0000-0001}
\affiliation{%
  \institution{Institution2}
  \streetaddress{Street2}
  \city{City2}
  \state{State2}
  \country{Country2}
  \postcode{Zip2}
}
\email{znowicki@institution2.edu}

\begin{document}

\maketitle

\section{Introduction}
The above diagram is a motivating example.

\section{DPL Syntax}
The above diagram shows but one sample of a practical DPL syntax. Variations and improvements on this syntax can be imagined. The above syntax is being used to produce actual applications like term-rewriting (t2t text-to-text rewriting) compilers, LLMs, DSLs for creating DSLs, Visual Shell prototypes, games, etc.

This DPL syntax consists of only a few kinds of closed figures plus arrows plus text belonging to the closed figures. Everything else is considered to be a comment, and, is ignored. For example, the bold text ``Sequential Routing'' is ignored. Colors are ignored. Line shapes and line widths are ignored, and so on.

This diagram was drawn using the off-the-shelf diagram editor draw.io\footnote{\url{https://www.draw.io}}. The editor saves the diagram in a modified form of XML, called graphML\footnote{\url{https://en.wikipedia.org/wiki/GraphML}}.

\end{document}
